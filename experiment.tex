\chapter[]{Anomaly Detection Using Recurrent Neural Networks}


\section{Introduction}

Chapters \ref{ch:ad} and \ref{ch:rnn}, separately, introduced anomaly detection in time series and recurrent neural networks as time series modelers.
%
This chapter outlines a procedure for finding anomalies using RNNs with the goal of mitigating many of the problems associated with anomaly detection that were discussed in \ref{ch:ad}.
%
As much as possible, the same procedure is applied for each time series used to test the procedure.


The outlines describes:
%
\begin{description}
%
\item[sampling:] the time series used for training and its associated sampling process
%
\item[recurrent autoencoders:] the specific form of the RNN
%
\item[training:] the training algorithm used on the RNN
%
\item[Bayesian optimization:] the search for optimized parameters for the procedure
%
\end{description}



\section{Sampling: Sliding Windows}
\label{sec:sampling}

Some univariate time series \emph{with} anomalies were chosen or generated to test the anomaly detection process:

\begin{description}

\item[electrocardiogram  (ECG):] a signal from the human heart taken from the PhysioNet \cite{PhysioNet} database.

\item[polysomnography ECG (PSG-ECG):] human ECG data \cite{PhysioNet} taken during a sleep study%
\footnote{University College Dublin Sleep Apnea Database, \url{http://www.physionet.org/physiobank/database/ucddb/}, doi:10.13026/C26C7D, study id. ucddb002}%
.

\item[power demand:] a building's electrical power usage data
\footnote{\url{http://www.cs.ucr.edu/~eamonn/discords/power_data.txt}} referenced in the influencial HOT-SAX paper \cite{Keogh2005} (for anomaly detection in time series).

\item[spike:] a generated sequence that is simply evenly-spaced `spikes' of the same height.

\item[sine:] a generated sinusoidal signal.

\end{description}
\noindent
%
The normal behaviour, as well as anomalous behaviour, of these time series can be seen in the plots presented in Section \ref{sec:results}.

While there is only one `master' time series to train on (in each test), the RNN needs to see many `samples' from the master time series.
%
The sliding window procedure, described in \ref{sec:adsample}, is used to get samples of some window length.
%
Additionally, since RNNs do not need a fixed window, sliding windows are obtained for more than one window length.
%
Furthermore, the samples are collected into `mini-batches' (of the same window length) to be more compatible with the training procedure.
%
The window length is incremented (size skip) from some minimum until it is not possible to obtain a mini-batch (so the largest window will be close to the length of the time series).


Table \ref{tbl:winspec} specifies the relevent sizes and increments for the sampling process.
%
The values were adjusted manually until a `reasonable' number of samples were found.
%
However, the minimum window length was chosen such that meaningful dynamics were captured (regardless of the scale of the anomaly).


\begin{table}[H]
  \centering
  \csvreader[
  % need p b/c the below \ifthenelse 'filters' make the tbl cell ugly
  tabular=|p{.75in}||c|c|c|c|c||c| 
  ,table head=
  \hline
  \bfseries series
  &  length 
  &  min. win.
  &  slide skip 
  &  size skip 
  &  batch size
  &  $\Rightarrow$ \bfseries samples 
  \\ \hline  \hline
  ,late after line=\\\hline
  ]%
  {tbls/sampling.csv}%
  {
    batch_size=\bs
    ,length=\ln
    ,min_winsize=\mw
    ,name=\nm
    ,nsamples=\ns
    ,slide_jump=\ss
    ,winsize_jump=\wss
  }%
  {
    %these act like filters. so you have to explicitly get them
    \ifthenelse{ \equal{\nm}{sin}}{   sine  }{}
    \ifthenelse{ \equal{\nm}{ecg}}{   ECG  }{}
    \ifthenelse{ \equal{\nm}{power}}{ power  }{}
    \ifthenelse{ \equal{\nm}{sleep}}{ PSG-ECG}{}
    \ifthenelse{ \equal{\nm}{spike}}{ spike  }{}
    & \ln & \mw & \ss & \wss & \bs & \ns
  }
\caption[]{Time series sample specifications} %todo: what case?
\label{tbl:winspec}
\end{table}


\section{RNN Setup: Autoencoder}

The RNN needs to learn what normal time series behaviour is.
%
So an autoencoder is used which can learn expected behavior by setting the target, $\vc{y}$, to be the input, $\vc{x}$.
%
The loss function is MSE (Equation \ref{eqn:mse}).
%
Furthermore, to prevent the RNN from learning trivial identity functions, Gaussian noise is added to the input where the standard deviation is equal to .75 the standard deviation of the whole time series.

\begin{equation*}
 \tilde{\x} = \x
 + \mathcal{N}(0,(0.75\sigma_{\mathrm{std}}(\x))^2)
\end{equation*}
\noindent
%
Note the comparison in the loss function is between the (uncorrupted) signal, $\x$, and the output from the network, $\vc{o}$, given $\x$: $L(\vc{o}(\tilde{\x}),\x)$.
%
With this setup, a denoising autoencoder, the data generating process, $p(\x)$, is implictly learned \cite{Bengio2013}.


\section{Training}


SGD with RMSprop \cite{Tieleman2012} for parameter updates has been demonstrated to provide results that are similar to more sophisticated second-order methods but with significantly less computational cost \cite{Dauphin}.
%
Another benefit of RMSprop is that it is designed to work with mini-batch learning.
%
Since the training data is highly redunant (from sliding windows), it is expected that computing the gradient (to update parameters) from a subset of the data (mini-batch) is more efficient than computing the gradient for all the data.


For each derivative in the gradient, RMSprop keeps an exponentially-weighted moving average of derivative magnitudes which normalizes the derivative by its root-mean-squared.
%
More specifically, the (overall) RNN training procedure is outlined in the following box.

\begin{algorithm}{Training Procedure}
\begin{algorithmic}
\STATE \STATE

\STATE \COMMENT{obtain mini-batches (as Section \ref{sec:sampling})}
\STATE $\mathcal{X} \gets \mathrm{sampling}(\x)$
\STATE \STATE

\STATE \COMMENT{randomly split mini-batches into training and validation sets}
\STATE $\mathcal{X}_t \gets \mathrm{choose(75\%,\mathcal{X})}$
\STATE $\mathcal{X}_v \gets \mathcal{X} - \mathcal{X}_t$
\STATE \STATE

\STATE \COMMENT{initalize}
\STATE $\vc{\theta} \gets \vc{0}$
\COMMENT{initial RNN parameters are 0}
\STATE \COMMENT{set training parameters}
\STATE $\alpha = 10^{-4}$
\COMMENT{learning rate}
\STATE $h = 14$
\COMMENT {RMS `halflife'}
\STATE $\gamma \gets e^{\frac{-\ln{2}}{h}}$
\STATE $r = 10^{-8}$
\COMMENT{RMS regularizer}
\STATE patience = 5
\COMMENT{stopping criterion parameter}
\STATE min\_improvement = $.005$
\COMMENT{stopping criterion parameter}
\STATE \STATE

%\STATE $i \gets 1$
\FOR{epoch $i$}
\STATE{ $\mathcal{X}_t \gets  \mathrm{shuffle}(\mathcal{X}_t)$}

\FOR{minibatch $M$ in $\mathcal{X}_t$ }

\FOR{parameter $p$ in $\vc{\theta}$}
\STATE{} \COMMENT{per-parameter, update $p$ according to RMSprop \cite{Graves2013b}
(indices to $p$ omitted)}
\STATE{
  \begin{flalign*}
    f_{i+1}
    &\gets 
    \gamma f_i 
    +
    (1 - \gamma) 
    \frac{\partial{L}}{\partial{p}} && 
    \\
    g_{i+1}
    &\gets
    \gamma g_i
    +
    (1 - \gamma)
    \left(
      \frac{\partial{L}}{\partial{p}}
    \right)^2 && 
    \\
    p_{i+1}
    &\gets 
    p_i
    - 
    \frac{\alpha}{
      \sqrt{g_{t+1} 
        - f_{t+1}^2 
        + r}
    }
    \frac{\partial{L}}{\partial{p}} &&
  \end{flalign*}
}
\ENDFOR
\ENDFOR

\STATE \COMMENT{compute average loss on validation set}
\STATE{
  $
  v_i
  \gets
  \frac{1}{|\mathcal{X}_v|}
  \sum_{\x_v \in \mathcal{X}_v}
  L(\x_v,\vc{o})
  $
}


\STATE \COMMENT{stop when no improvement more than patience times}
\IF{
$v_{min}-v_{i}> v_{min} \cdot \mathrm{min\_improvement}$
}
\STATE{$v_{min} \gets v_i$}
\STATE{$i_{min} \gets i$}
\ENDIF

\IF{$i - i_{min} > \mathrm{patience}$}
\STATE{STOP} \COMMENT{return $\vc{\theta}$}
\ENDIF

\STATE{$i \gets i+1$}
\ENDFOR

%\STATE% \STATE
\end{algorithmic}
\end{algorithm}

%todo: use s=h in lstm

The \texttt{theanets} \cite{Johnson2015} implementation of RNNs and RMSprop
\footnote{With every calculation of $\vc{o}$, the RNN states are initialized to 0.}
was used.
%
\texttt{theanets} is based on the mathematical expression compiler, \texttt{theano} \cite{Bergstra2010}.
%
Gradients were computed using \texttt{theano}'s automatic differentiation feature instead of explicitly-defined backpropagation \cite{Rumelhart1986}.

%\section{Hyper-parameter Optimization}
%In the previous section, 

While the training procedure optimizes $\vc{\theta}$, there are other parameters that could be adjusted to minimize the loss.
%
The number of layers, $n_l$, and the `size' of each layer, $n_s$, corresponding the dimension of vector $\vc{s}$ (which equals the cell state mem), were chosen as `hyper-parameters' for optimization in a Bayesian optimization process.
%
Bayesian optimization is suited for optimizing RNN training because
1) it tries to minimize the number of (expensive) objective function calls, which, in this case, is the training procedure, and
2) it considers the stochasticity of the function, which, in this case, the selection of training and validation data are random in addition to the training data shuffling on each epoch.


The \texttt{spearmint} \cite{snoek2012practical} package was used to drive the hyper-parameter optimization process.
%
The process was programmed to save RNN parameters after every optimization iteration.
%
\texttt{spearmint} takes a parameter search space and a maximum number of optimization iterations.
%
However, the process was stopped until the RNN with the minimum validation loss was subjectively able to detect anomalies
%
\footnote{Further optimization was possible with increasing RNN size but loss reduction diminished relative to the associated increase in computational expense. See Bayesian optimization figures in Section \ref{sec:results} }
.



\section{Results}
\label{sec:results}



%todo. include tbls and pdfs in repo in final

%%% Local Variables:
%%% mode: latex
%%% TeX-master: "thesis"
%%% End:
