\chapter{Reproducing Results}
\label{ch:reproduce}



\section{Introduction}


This chapter shows the commands that were issued to obtain the results shown in Chapter \ref{ch:results}.
% 
The commands build on the infrastructure setup explained in Chapter \ref{ch:pcc}.
% 
Also, the commands (themselves) specifically address requirements of the Bayesian optimization process driven by \textsf{Spearmint} as described in \ref{sec:pccmotivation}.
% 
As such they are the highest-level commands that can be automated as well but were not to give the investigator some control over execution.
% 
But the control is only operational as similar results should be obtained regardless of how the computations are executed.


However, first, a set of `manual' instructions are provided as well that are more typical of instructions accompanying computational publications (if at all!).
% 
The manual instructions, in contrast to the automated instructions, are not straightforward to extend for distributed execution.
% 
Nonetheless, the automated execution is just encapsulating the manual procedure with an automated one which embodies a distributed environment.
%
So, becoming familiar with the manual process helps with understanding the automated process.


In any case, all software used was current as of January, 2016.
% 
So, if the software version is not identified, it can be assumed that it was current as of January 2016.
% 
So newer software \emph{may} work.



\section{Manual Execution}
\label{sec:manual}


\begin{description}

\item[0. Install prerequisite software.] \hfill

  Install the following.
  % 
  Using the versions indicated ensure reproduction of results
  \footnote{
    Each has its own, possibly multiple, methods of installation!
    % 
    Furthermore, the listed software can have their own prerequisites possibly with version restrictions.
    % 
    Therefore, using \textsf{Conda} from Continuum Analytics is recommended for installing \textsf{Python}-based computational software.
    % 
    Conda can be obtained from \url{http://conda.pydata.org/miniconda.html}.
  }%
  :
  
  \begin{tabular}{ll}
    \textsf{Python}:              & 2.7.11 \\
    \textsf{NumPy}:               & 1.10.1 \\
    \textsf{SciPy}:               & 0.16.0 \\
    \textsf{scikit-learn}:        & 0.16.1 \\
    \textsf{MongoDB}:             & 3.1.9 \\
    \textsf{PyMongo}:             & 2.8 \\
    \textsf{psutil}:              & 3.2 \\
    modified \textsf{Spearmint}:  & \textsf{Git} commit ccc503ae08798cb5ed9fd6090310de89b6d9b39f \\
                                  & from the repository at \\ 
                                  & \url{https://github.com/majidaldo/Spearmint} \\
    \textsf{Theano}:              & \textsf{Git} commit fe58ada7cbf8b6fd031d9ad9b3c6c570b1717f9b \\
                                  & from the official repository at \\
                                  & \url{https://github.com/Theano/Theano} \\
    \textsf{Theanets}:            & \textsf{Git} commit ae588cfd6b7b04c02a603ecfee4ba14c68d46ca2 \\
                                  & from the official repository at \\ 
                                  & \url{https://github.com/lmjohns3/theanets}
  \end{tabular}



\item[1. Obtain main program code] \hfill

  The main computational code is available at \url{https://github.com/majidaldo/tsad}.
  % 
  Use \textsf{Git} commit 37703f3d10cf5ff26cc72cc8b7dff639cffda78c.
  % todo: where is the doi num?
  


\item[2. Setup] \hfill

  \begin{description}

  \item[1. Configure RNN storage.] \hfill

    With local execution of \textsf{MongoDB}, the server host must be set to \texttt{localhost} in \path{config.yml} in the program code.

\begin{minted}{yaml}
---
rnndb: localhost
\end{minted}


  \item[2. Start \textsf{MongoDB}.] \hfill
    
    Usually, this involves some kind of invocation of the \mintinline[]{bash}{mongod} command.
    % 
    Refer to \textsf{MongoDB} documentation for specific instructions.

    The \textsf{MongoDB} instance will store RNN parameters and \textsf{Spearmint} progress.


  \item[3. Change to sequence directory.] \hfill

    Each sequence optimization is associated with two executables and a \textsf{Spearmint} configuration file in the directory \path{tsad/experiments}.
    % 
    The correspondence between the names used in Section \ref{sec:sampling} to the directory names is as follows:
    
    \begin{tabular}{ll}
      spike-1 & \texttt{spikelv} \\
      spike-2 & \texttt{spikereg} \\
      sine    & \texttt{sin} \\
      power   & \texttt{power} \\
      ECG     & \texttt{ECG} \\
      ECG-PSG & \texttt{sleep} \\
    \end{tabular}


  \item[4. Configure \textsf{Spearmint}] \hfill
    
    Ensure that \textsf{Spearmint} runs the non-distributed executable, \path{o.py}, by modifying the contents of the configuration file, \path{config.json}, if needed.
    % 
    \path{"o.py"} should be the value for \texttt{main-file} attribute of the configuration file.
    %
    Also, the \texttt{address} attribute for \texttt{database} should be set to \texttt{"localhost"} since it is assumed that a local \textsf{MongoDB} is running.
    %
    With these settings, \path{config.json} should be as follows.

\begin{minted}[]{json}
{

    "experiment-name": "power",
    "database": {"address":"localhost"},
    "language"   : "PYTHON",
    
    "resources" : {
	"my-machine" : {
	    "scheduler"         : "local",
	    "max-concurrent"    : 1,
	    "max-finished-jobs" : 20
	}
    },

    "tasks": {
	"main" : {
	    "type"       : "OBJECTIVE",
	    "likelihood" : "GAUSSIAN",
	    "main-file"  : "o.py",
	    "resources"  : ["my-machine"]
	}
    },
    
    "variables" : {
	"nl" : {
	    "type" : "INT",
	    "size" : 1,
	    "min"  : 1,
	    "max"  : 2
	},
	"n" : {
	    "type" : "INT",
	    "size" : 1,
	    "min"  : 1,
	    "max"  : 10
	},
	"iter" : {
	    "type" : "FLOAT",
	    "size" : 1,
	    "min"  : 0,
	    "max"  : 1
	}
    }
}
\end{minted}

    In addition, the paramater space for the Bayesian optimization search can be set by changing the \texttt{min} and \texttt{max} under the \texttt{variables} attribute.
    % 
    \texttt{nl} is the variable name for number of RNN layers, $l$, while \texttt{n} is the number of nodes in a layer, $n$.



\end{description}

\item[3. Run \textsf{Spearmint}.] \hfill

\begin{minted}{bash}
python /path/to/spearmint/main.py .    
\end{minted}


\end{description}


\section[Automated Execution]{Automated Execution%
\protect\footnotemark[1]
}
\footnote{
Unfortunately the reproducibility of the automated computation is flawed because the most recent version of the software identified in Step 0 of the manual execution instructions is installed by default.
%
However, the versions can be fixed easily by modifying the appropriate command in the file \texttt{docker/0100-computer/Dockerfile} in the automation code for CPU operation or \texttt{docker/01111-computer-gpu/Dockerfile} for GPU operation.
}

%unfortunately in the interest of automating the latest
%unfortunately s/w vers not thourougly integrated. but refer to manual if any problem. auto just gets the latest. instructions in dockerfiles. but were encoded when ver became an issue.
%sometimes new s/w good for performance reasons. best to work wtih latest. as developed over months.
%most are in the computer dockerfile easily encoded

\begin{description}



\item[1. Obtain automation code.] \hfill

  The code is available at \url{https://github.com/majidaldo/tsad-sys}.
  % 
  It is made of several submodules so \texttt{git} should be used with the \texttt{recursive} option.%
  
  %
\begin{minted}{bash}
git clone --recursive https://github.com/majidaldo/tsad-sys
\end{minted}
  % 
  Then, change into the code directory and checkout the appropriate version.
  

\begin{minted}{bash}
git checkout 8f20b9768cd4c4ae83bbb8e156e5d7aa9fbb2565
\end{minted}

  The automation code also contains the computational code as a submodule in the \texttt{tsad} directory.


\item[2. Setup] \hfill

\begin{description}

\item[1. Setup automation.] \hfill

  Follow instructions according to the \path{README} file \url{https://github.com/majidaldo/personal-compute-cloud/tree/thesis0/README.md} to initialize the automation, provision the \texttt{init} machine, and provision compute machines.
  % 
  There is no need to change any system variable set in the automation code.


\item[2. Start services.] \hfill

  \texttt{ssh} into the \texttt{init} machine to run services on it.
  % 
  The services (and compute applications) are all \textsf{Docker} applications.
  % 
  Each application corresponds to a directory within the \texttt{docker} directory in the automation code.
  % 
  The application directories contain shell scripts to run the application.

  \begin{description}

  \item[1. Start \textsf{MongoDB}.] \hfill

    Change into the \path{/project/docker/1111-mongodb} directory and execute \mintinline{bash}{./run.sh}.

  \item[2. Start the \textsf{IPython} controller.] \hfill

    Similarly, change into the \path{/project/docker/0200-ipycontroller} directory and execute \mintinline{bash}{./run.sh}.

    The \textsf{IPython} controller handles the distribution of computational tasks across workers.
    % 
    Its data is configured for storage in \texttt{MongoDB} as well.

  \end{description}


\item[3. Configure RNN storage.] \hfill
  
  Just like in manual execution, the RNNs are configured for storage in a \textsf{MongoDB} database.
  %
  So similarly, confirm that the database host is set to \texttt{mongodb} in \path{tsad/config.yml}.


\item[4. Configure \textsf{Spearmint}] \hfill
    
  Again, like in manual execution, change into a sequence directory in \path{tsad/experiments}.
  %
  Ensure that \textsf{Spearmint} runs the \emph{distributed} executable, \texttt{po.py}, by modifying the contents of the configuration file, \path{config.json}, if needed.
    % 
    \path{po.py} should be in the \texttt{main-file} attribute of the configuration.
    %
    Furthermore, set the number of concurrent compute workers \textsf{Spearmint} controls by changing the value of the \texttt{max-concurrent} attribute.


    %
    Also, the \texttt{address} attribute for \texttt{database} should be set to \texttt{"mongodb"} since it is assumed that a \emph{remote} instance of \textsf{MongoDB} is running.
    %
    With these settings, \path{config.json} should be as follows.

\begin{minted}[]{json}
{

    "experiment-name": "power",
    "database": {"address":"mongodb"},
    "language"   : "PYTHON",
    
    "resources" : {
	"my-machine" : {
	    "scheduler"         : "local",
	    "max-concurrent"    : 5,
	    "max-finished-jobs" : 20
	}
    },

    "tasks": {
	"main" : {
	    "type"       : "OBJECTIVE",
	    "likelihood" : "GAUSSIAN",
	    "main-file"  : "po.py",
	    "resources"  : ["my-machine"]
	}
    },
    
    "variables" : {
	"nl" : {
	    "type" : "INT",
	    "size" : 1,
	    "min"  : 1,
	    "max"  : 2
	},
	"n" : {
	    "type" : "INT",
	    "size" : 1,
	    "min"  : 1,
	    "max"  : 10
	},
	"iter" : {
	    "type" : "FLOAT",
	    "size" : 1,
	    "min"  : 0,
	    "max"  : 1
	}
    }
}
\end{minted}

    In addition, the paramater space for the Bayesian optimization search can be set in the same manner described in manual execution.
  

\end{description}

\item[3. Run Spearmint.] \hfill

  In distributed operation, \textsf{Spearmint} is run as a coordinating `service' so it is recommended that the following steps are run on the \texttt{init} machine.
 
\begin{description}


\item[1. Enter compute environment.] \hfill

  Change into the \texttt{docker} directory that represents the computing environment, \path{/project/docker/0100-computer}.
  %
  Then, start the environment with the following command%
  \footnote{This runs an \textsf{IPython} kernel but it is irrelevant.}%
  .

\begin{minted}{bash}
./run.sh ipykernel
\end{minted}
%
Now, enter the environment by issuing the following \textsf{Docker} command.

%
\begin{minted}{bash}
docker exec -ti init-ipykernel-1 bash
\end{minted}
%
Notice that the command includes \texttt{init} because the \path{run.sh} command assigns a name to the \textsf{Docker} container that includes the hostname.


\item[2. Execute.] \hfill

  Now inside the \textsf{Docker} container, change into the sequence directory in \path{/data/experiments} and execute the following.

  %
\begin{minted}{bash}
python ~/spearmint/spearmint/main.py .
\end{minted}


\end{description}


\item[4. Join compute workers.] \hfill

The workers are \textsf{IPython} engines that connect to the \textsf{IPython} controller.
%
Running compute workers on compute machines provisioned by the automation is straightforward. 
%
However, other workers can join as long as their environment is the same as that described in the manual execution section.

\begin{description}


\item[Join automated workers.] \hfill

  \texttt{ssh} into a compute worker and run the following in the \path{/project/docker/0100-computer} directory.

\begin{minted}{bash}
./run.sh ipyengine 1
\end{minted}

%
The 1 indicates that just one worker will be started but the number can be increased.


A NVIDIA GPU-equipped machine can start a GPU accelerated worker by running the same command in the \path{/project/docker/0111-computer-gpu} directory.


Computation progress can be monitored by issuing a \texttt{docker logs} command for the container name.

\begin{minted}{bash}
docker logs compute-machine-name-ipyengine-1
\end{minted}

The 1 stands for the first worker started from the previous \texttt{./run.sh} command.
%
The container name can be found from the list output from the command, \mintinline{bash}{docker ps}.

\item[Manually join worker.] \hfill

An \texttt{IPython} engine can join the group of workers if it establishes connectivity to the \textsf{MongoDB} database and the \textsf{IPython} controller.
%
Assuming the worker is on the local machine setup as per the manual execution instructions in Section \ref{sec:manual}, connect the worker as follows.

\begin{description}[style=unboxed]


\item[1. Forward \textsf{MongoDB} and \textsf{IPython} controller network ports to \texttt{init} computer.] \hfill

  Instead of communicating to local services, local communication is forwarded to services runninig on the \texttt{init} machine.
  % 
  Since the ports are forwarded from the local machine, in the main program code directory (say, \texttt{tsad}) confirm that the RNN code is communicating with \textsf{MongoDB} as if it were local;
  %
  check the contents file \path{config.json} so that it is as follows.

\begin{minted}{yaml}
---
rnndb: localhost
\end{minted}

  %
  Then, also in the \texttt{tsad} directory, navigate to the \texttt{docker} directory.
  %
  In the \texttt{docker} directory, the \path{port.sh} script forwards network ports.
  % 
  Forward the \textsf{MongoDB} port with \mintinline{bash}{./port.sh 27017}.
  %
  In another command console, forward (one of) the \textsf{IPython} controller ports with \mintinline{bash}{./port.sh 4321}.
  %
  Keep the consoles open.


\item[2. Start \textsf{IPython} engine.] \hfill

  From the \texttt{tsad} directory, start the \textsf{IPython} engine as follows.

\begin{minted}{bash}
ipengine \
--file=/path/to/tsad-sys/files/\
.ipython/profile_default/security/ipcontroller-engine.json \
--ssh="core@init"\
--sshkey=/path/to/.vagrant.d/insecure_private_key
\end{minted}

%
The \texttt{tsad-sys} directory is the automation code directory.
%
In this command, the \textsf{IPython} engine is making a \textsf{SSH} network tunnel to the \textsf{IPython} controller residing in the \texttt{init} machine.
%
The \textsf{IPython} controller writes its (dynamic) network configuration to the \texttt{ipcontroller-engine.json} file on its start.
%
This information is available on the (outside) local machine through the file share.
%
For the location of the \path{insecure\_private\_key} file, check your \textsf{Vagrant} installation.

\end{description}
  
\end{description}

\end{description}




\section{Reproduction of Figures}


The figures shown in Chapter \ref{ch:results} are produced in a separate process associated with the production of this document.
%
Figure production is not fully automated but the previously described computation environments can used as a basis for producing the figures by following the steps below.
% 

\begin{description}[style=unboxed]


\item[1. Install data analysis and plot tools.] \hfill

  Install the following in a compute environment:

  \begin{tabular}{ll}
    \textsf{matplotlib} & 1.5.1 \\
    \textsf{Seaborn} &  0.6.0 \\
    \textsf{pandas} & 0.17.1
  \end{tabular}


\item[2. Obtain code for this document.] \hfill

  Download the code from \url{https://github.com/majidaldo/tsad-docs}.
  % 
  Then checkout the appropriate version as follows.

\begin{minted}{bash}
git checkout c524238ba7a4e514e6cb155afe42c3b4808048ed
\end{minted}


\item[3. Configure location of main computaion code.] \hfill

  In the document code, change the path in \path{figs/tsad.py} to the location of the main computation code.


\item[4. Generate figures.] \hfill

  Assuming \textsf{MongoDB} is running, from the \texttt{figs} directory execute the following command.

\begin{minted}{bash}
python figs.py all
\end{minted}

The figures will be produced as files written in the same directory.

\end{description}




%%% Local Variables:
%%% mode: latex
%%% TeX-command-extra-options: "-shell-escape"
%%% TeX-master: "../thesis"
%%% End:
